(e). (3 points) \textsl{In real life, the police need to effectively use all the information they have gathered, to identify who is responsible for running the illegal activities of the group. Armed with a qualitative understanding of the centrality metrics from Part (d) and the quantitative analysis from part Part (b) Question 5, integrate and interpret the information you have to identify which players were most central (or important) to the operation. ($\sim$100 words, 200 word limit.)}\\

\textbf{Solution}:\\
The comparison of different centrality mean values of all phases for all nodes represent the phase-average importance of individual nodes within the CAVIAR network (see table \ref{tab:centrality_important_nodes}). The importance of a node can be defined as the power to connect other nodes which increases each individuals influence.\\

\begin{table}[h]
	\centering
	\begin{tabular}{|c|c|c|c|}
		\hline 
		Centrality type & Highest & Second highest & Third highest \\ 
		\hline 
		Betweenness & no. 1 (0.66) & no. 12 (0.17) & no. 3 (0.13) \\ 
		\hline 
		Eigenvector & no. 1 (0.55) & no. 3 (0.30) & no. 85 (0.19)\\ 
		\hline 
	\end{tabular} 
	\caption{Comparison between different centrality measures and corresponding most important nodes (individuals of CAVIAR network, mean values over all phases)}
	\label{tab:centrality_important_nodes}
\end{table}

Both, mean betweenness and mean eigenvector centrality, state that node no. 1 is the most important node. As mentioned before, the betweenness centrality is a good indicator to judge on which individual the police should focus on. Therefore, the nodes no. 12 and 3 are the next two nodes, which should be persecuted.\\